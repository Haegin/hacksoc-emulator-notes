\documentclass[12pt,a4paper]{article}
\pagestyle{headings}

\usepackage[utf8x]{inputenc}
\usepackage{amsmath}
\usepackage{amsfonts}
\usepackage{amssymb}
\usepackage{amsthm}
\usepackage{graphicx}
\usepackage[margin=0.9in]{geometry}
\usepackage{ucs}
\usepackage[british]{babel}
\usepackage[nodayofweek]{datetime}
\usepackage{enumitem}
\usepackage{multirow}
\usepackage{tabularx}
\usepackage{float}
\usepackage{listings}
\usepackage{pdflscape}
\usepackage[usenames,dvipsnames]{color}
\usepackage{cite}

\lstset{basicstyle=\small\ttfamily}
\lstset{showstringspaces=false}
\lstset{numbers=left, numberstyle=\tiny, stepnumber=1, numbersep=5pt}
\lstset{keywordstyle=\color{MidnightBlue}\bfseries}
\lstset{commentstyle=\color{JungleGreen}}
\lstset{identifierstyle=\color{OliveGreen}}
\lstset{stringstyle=\color{Red}}
\lstset{backgroundcolor=\color{LightGray}}
\lstset{breaklines=true}

\definecolor{LightGray}{rgb}{0.9,0.9,0.9}

\floatstyle{boxed}
\restylefloat{figure}

\author{Michael Walker}
\title{Lecture 1. Introduction}
\date{}

\begin{document}

\maketitle{}

This is intended to be a \textbf{very} brief introductory lecture,
where I'll introduce the terminology and the tools that we'll be
using.

\tableofcontents

\pagebreak
\section{The Project}

So, you want to learn how to program? Well, the best way to learn is
to do it! In this ``module'' we'll be making an emulator for the
PicoBlaze microprocessor from scratch in Python.

If you're currently a first year doing ICAR and TPOP, this should
complement those modules nicely.

\pagebreak
\subsection{Components of a Computer}

A computer is made up of a lot of pieces, but for our purposes, the
three we're most interested in are the \textbf{CPU}, \textbf{ALU}, and
\textbf{memory}.

The ALU is typically a component of the CPU rather than something
separate, so we'll usually refer to both as just the CPU.

Every computer needs at least a CPU and a memory to function. Without
a CPU it can't actually do any computing, and without a memory, it
can't store any programs to compute with.

\subsubsection{CPU / ALU}

The CPU is the bit which reads in an instruction from memory (fetch),
figures out what it is (decode), and does it (execute). And it just
does that all the time.

The ALU is a component of the CPU which deals with arithmetical and
logical operations, such as addition and bit-shifting. We're not going
to separate them in our implementation, but this might be useful in
implementing more complex processors.

So, how can we implement a CPU? Well, since it's a solid object, a
class may be a good start:

\lstinputlisting[language=python]{cpu1.py}

\subsubsection{Memory}

\pagebreak
\subsection{The PicoBlaze}

\pagebreak
\section{Source control: Being able to fix mistakes.}

\pagebreak
\section{Testing: How to make sure it's right.}

\end{document}