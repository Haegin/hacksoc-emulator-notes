\documentclass[12pt,a4paper]{article}
\pagestyle{headings}

\usepackage[utf8x]{inputenc}
\usepackage{amsmath}
\usepackage{amsfonts}
\usepackage{amssymb}
\usepackage{amsthm}
\usepackage{graphicx}
\usepackage[margin=0.9in]{geometry}
\usepackage{ucs}
\usepackage[british]{babel}
\usepackage[nodayofweek]{datetime}
\usepackage{enumitem}
\usepackage{multirow}
\usepackage{tabularx}
\usepackage{float}
\usepackage{listings}
\usepackage{pdflscape}
\usepackage[usenames,dvipsnames]{color}
\usepackage{cite}
\usepackage{hyperref}

\lstset{basicstyle=\small\ttfamily}
\lstset{showstringspaces=false}
\lstset{numbers=left, numberstyle=\tiny, stepnumber=1, numbersep=5pt}
\lstset{keywordstyle=\color{MidnightBlue}\bfseries}
\lstset{commentstyle=\color{JungleGreen}}
\lstset{identifierstyle=\color{OliveGreen}}
\lstset{stringstyle=\color{Red}}
\lstset{backgroundcolor=\color{LightGray}}
\lstset{breaklines=true}

\definecolor{LightGray}{rgb}{0.9,0.9,0.9}

\floatstyle{boxed}
\restylefloat{figure}

\author{Michael Walker}
\title{Lecture 1. Introduction}
\date{}

\begin{document}

\maketitle{}

This is intended to be a \textbf{very} brief introductory lecture,
where I'll introduce the terminology and the tools that we'll be
using.

\tableofcontents

\pagebreak
\section{The Project}

So, you want to learn how to program? Well, the best way to learn is
to do it! In this ``module'' we'll be making an emulator for the
PicoBlaze microprocessor from scratch in Python.

If you're currently a first year doing ICAR and TPOP, this should
complement those modules nicely.

\pagebreak
\subsection{Components of a Computer}

A computer is made up of a lot of pieces, but for our purposes, the
three we're most interested in are the \textbf{CPU}, \textbf{ALU}, and
\textbf{memory}.

The ALU is typically a component of the CPU rather than something
separate, so we'll usually refer to both as just the CPU.

Every computer needs at least a CPU and a memory to function. Without
a CPU it can't actually do any computing, and without a memory, it
can't store any programs to compute with.

\subsubsection{CPU / ALU}

The CPU is the bit which reads in an instruction from memory (fetch),
figures out what it is (decode), and does it (execute). And it just
does that all the time.

The ALU is a component of the CPU which deals with arithmetical and
logical operations, such as addition and bit-shifting. We're not going
to separate them in our implementation, but this might be useful in
implementing more complex processors.

So, how can we implement a CPU? Well, since it's a solid object, a
class may be a good start. A CPU also has some flags and registers,
so let's add a way of storing those:

\lstinputlisting[language=python]{cpu1.py}

\pagebreak
\subsubsection{Memory}

So, memory. What \textbf{is} memory? Well, it's a big block of space
where you can store stuff. Each location in memory has a unique number
associated with it. These numbers count up from 0.

In other words, it's like a big array. Let's implement it in exactly
that way.

\lstinputlisting[language=python]{mem1.py}

\pagebreak
\subsection{The PicoBlaze}

The PicoBlaze is a processor by Xilinx, it's a very minimal
architecture designed for implementation on an FPGA. The memory
consists of 256 8-bit values (``words''), and each instruction is 2
words. It has 16 registers and two flags (``zero'' and ``carry'').

Let's extend our CPU design to have this:

\lstinputlisting[language=python]{cpu2.py}

The PicoBlaze instruction set can be divided into six rough groups, of
which you'll be implementing four (unless you do the extra
work). In order of our implementation, these are:

\begin{description}
  \item[Logic] Logical operations on the contents of the registers,
    and also loading values into registers.

  \item[Arithmetic] Mathematical operations (with and without carry)
    on the contents of registers.

  \item[Shift/Rotate] Bitwise shifting on register contents

  \item[Control] Jumping and subroutines. Implementing these
    requires implementing a call stack.

  \item[Input/Output] Reading and writing values to and from locations
    other than the memory.

  \item[Interrupts] Enabling and disabling interrupts, external events
    than can trigger special behaviour in the processor.
\end{description}

\pagebreak
\section{Source control: Being able to fix mistakes.}

It may seem ok to just have one copy of the code and work on it. If
you think you might want to undo something, you can always make a
back-up of it just in case.

But what if you don't \textbf{know} that you might want to undo
something? What if you only realise something is a problem after you
did it, without making a backup?

The answer, is source control.

With source control, you still only have one copy of the files, but
you tell a program every time you have made a change. This program
keeps track of all the changes and can undo them if need be.

\subsection{Using Git}

Git is a popular source-control (sometimes called version-control)
tool. Let's see how to use it.

Firstly, we need to set-up git for the current directory:

\begin{lstlisting}[language=bash]
git init
\end{lstlisting}

Well, that was easy. But now you need to tell it about changes. Let's
say you've modified files \texttt{foo}, \texttt{bar}, and
\texttt{baz}.

\begin{lstlisting}[language=bash]
git add foo bar baz
git commit
\end{lstlisting}

You'll then be prompted to enter a description of what you just did.

Files can also be deleted:

\begin{lstlisting}[language=bash]
git rm foo
\end{lstlisting}

Sometimes there are certain files you don't want to add. For that, you
can edit (or make, if it doesn't exist), a \texttt{.gitignore}
file. Let's say you want git to ignore all files that end with
\texttt{.o}, you might be writing some C for instance:

\begin{lstlisting}[language=bash]
echo `*.o' >> .gitignore
\end{lstlisting}

\subsection{Git Hosting}

The nice thing about git is that it's \textit{distributed}, there's no
one absolutely-in-control copy of your code, which all others are
subservient to. No. With git, you can have copies of your code in as
many places as you want, and you can just update them all to reflect
any changes at your leisure.

GitHub is a website that lets you do all of this. Let's say you have a
repository somewhere containing your code, and you want to be able to
use it. Well, firstly you need to tell git what it is and give it a
little name:

\begin{lstlisting}[language=bash]
git remote add MyAwesomeOnlineCode http://www.example.com/thecode
\end{lstlisting}

You can then \textit{push} to and \textit{pull} from that remote
repository. Pushing sends your changes over to it, and pulling fetches
its changes and makes them visible to you. These are pretty simple.

\begin{lstlisting}[language=bash]
git push MyAwesomeOnlineCode master
git pull MyAwesomeOnlineCode master
\end{lstlisting}

Simple! ``But wait, what is that \texttt{master}?'', I hear you cry!
Well, you see, git has \textit{branches}. A branch is a different
version of your code. By default, there is just one branch and it is
called ``master''.

\pagebreak
\section{Testing: How to make sure it's right.}

\pagebreak
\section{Conclusion}

\subsection{Use source control all the time!}

It may seem like a hassle. Having to tell a program about all of your
changes, every single time you make a change. And it is a hassle, at
first. But after a little while you become used to it, and the
usefulness of it far outweighs the hassle of starting out.

All of these slides are available on GitHub, as is the code:

\url{https://github.com/Barrucadu/hacksoc-emulator-notes}

\url{https://github.com/Barrucadu/hacksoc-emulator-code}

\subsection{Homework}

\begin{itemize}
  \item Finish the \texttt{Memory} implementation: add \texttt{read}
    and \texttt{write} functions that ensure the memory address is in
    the correct range (if it's too big it should wrap around), and
    ensure that the value being written is in range (an error should
    be given if it's not).

  \item Implement the ability to read in a PicoBlaze binary file to
    memory.
\end{itemize}

\end{document}